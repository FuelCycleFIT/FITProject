% #########################################################################################
% #########################################################################################
% #########################################################################################
\section{Framework and Description Of The FIT Project}

The FIT (Functionality Isolation Test) Project is an international effort
devoted to improve the fuel cycle tools confidence. This section aims to
precisely describe and to present the framework of the project.

% -----------------------------------------------------------------------------------------
\subsection{Nuclear fuel cycle dynamic simulation tool}

Fuel cycle simulation simulators~\cite{NEA2016} development started many years
ago by several research and engineering institutions or consulting firms for
many wide range of applications. In the case of research and engineering
institutions in charge of supporting the operated nuclear installations, fuel
cycle codes are used to facilitate and  optimize the industrial operations. For
research and engineering institutions concerned by energetic transition, fuel
cycle tools are used to study and analyze future trajectories for prospective
reflections on the electric component of the transition. Also, for institutions
in charge of educational and training purposes in the nuclear field, such tools
help understanding the physics mechanism that drive a nuclear fleet and can be
used as an educational support. Consulting firms develop and use those tool to
provide enlightened advices to politics. For those reasons, a fuel cycle tool
can be seen as a decision making tool since results and analyses are directly or
indirectly used by industrial and political worlds.

A fuel cycle simulation is based on a computer simulator used to model and
compute the evolution of isotopes of interest in nuclear facilities, strategic
stocks and waste disposal. An important effort in tool development concerns
physics modeling that aims to describe complex physics or industrial processes.
A nuclear fuel cycle is then a very complex system in which isotopes evolution
can be impacted by various parameters such as reactor technology deployment,
fuel reprocessing strategies,... A fuel cycle code computes radio-nuclides
evolution in all the nuclear facilities from the definition of a nuclear fleet.
The material evolution is estimated during the irradiation process in reactors
and during cooling phases is other facilities. Taking into account all the
physics phenomena and industrial practices requires a large effort in software
development. For this reason, fuel cycle tools includes many modeling 
simplifications.

Bias or uncertainties in fuel cycle outputs are difficult to quantify as
multiple level of simplifications are imposed by the complexity of the simulated
system. While comparison with operated nuclear fleets are complex, such a
validation processes are not possible for prospective simulations involving on
innovative reactors. Uncertainty or bias come from many sources. Nuclear data
uncertainty has an impact on reactor calculation output used to tune fuel cycle
codes~\cite{Krivtchik_2014}. Moreover, reactor models rely on simplified reactor
descriptions, implying a use of biased neutronic data in such simulators
\cite{Somaini_2017}.  At the simulation scale, simplifications are also
required, somethings without an accurate estimation of their impact on the
simulation. 

Some international effort \cite{NEA2016} focus on benchmarking fuel cycle
simulators comparing output metrics at the scale of an operated nuclear fleet.
Results produced in the framework of those works are decisive to test the
ability of multiple codes to be in agreement. Nevertheless, such comparison are
focused on aggregated data and deviations between simulators may be hard to
interpret. The FIT project is design complementarity with those international
benchmarks and aims to provide informations allowing to increase confidence in
the data produced by fuel cycle simulation tools.

% -----------------------------------------------------------------------------------------
\subsection{Goals and intended impact of the FIT project}

The FIT Project was initiated in 2017. It aims to improve the confidence in the
data produced by fuel cycle simulation tools. The first goal is to gather the
community of fuel cycle specialists around the question of the confidence in
simulation outputs. The second goal is to determine the optimum level of details
a fuel cycle simulator needs relative to the type of study and the required
confidence level. The project relies on the wide variety of fuel cycle
simulators with a large range of complexity level are developed.

Improving the ability to reproduce an operated nuclear fleet involves increasing
the complexity of the simulation tool by developing new functionalities. A fuel
cycle functionality is the translation of a physical or technical process,
related to nuclear facilities, into computer software language. The
Table\ref{Tab:Funct} lists a set of functionalities that could be implemented
in addition/replacement of the default model of the simulator.

\begin{table}[h]
\centering
\begin{tabular}{ |p{0.5\textwidth}|p{0.5\textwidth}| }
  \hline
  Reference & Functionality to develop 
  \\ \hline
  Fix fresh fuel composition for each fuel batch loading 
  & Update fuel fuel composition with regards of the available material
  composition 
  \\ \hline
  Averaged thermal-power over the cycles 
  & Dynamic thermal-power follow up
  \\ \hline
  Use of fixed/averaged macroscopic cross section for fuel depletion
  & Time/\gls{BU}-dependent cross sections
  \\ \hline
  Steady state reactor start (no starting batches)
  & Fuel starting batches modeling \\
  \hline
\end{tabular}
\label{Tab:Funct}
\caption{Examples of simplified/complex functionality couples.}
\end{table}


The FIT project consists in isolating a functionality impacts on fuel
cycle simulations. Each functionality impacts will be assessed using a set of
simple basic exercises specifically designed for this purpose, called
``functionality isolation``. Each exercise of a functionality isolation will be
focused on a subset of output metrics related to a type of problem covered by
the fuel cycle study.  

For instance, the total mass of plutonium or the isotopic composition of
plutonium will be used as output of interest for fuel cycle studies dealing with 
nuclear fuels recycling. Minor actinides production could be added to the
analyses to take into account fuel cycle studies concerned by radio-protection.
Once the choice of output of interest is made, the effect of the functionality
will be quantified by specific estimators computed with the functionality
enabled and the reference case. Each participant can submit a resolution for the
exercise and conclusions can be drawn according to the level of agreement
of participants.

With this methodology, the FIT project could provide informations about which
functionalities are required regarding a specific question and an associated
precision/confidence.  A fuel cycle simulation tool user starts from a technical
question, such as "In a \gls{PWR} fleet, considering spent \gls{UOX} fuel plutonium
reprocessing, what is the optimum \gls{PWR} \gls{UOX}-\gls{MOX} ratio that allow no/low plutonium
accumulation ?". The user then identify the set of output metrics required to
answer the technical question. In the example above, the user needs to assess
the plutonium inventory contained in facilities between the \gls{UOX} spent fuel and
the \gls{PWR} \gls{MOX} fuel. The user can then use the FIT project results to decide what
are the code required functionalities or which functionalities will produce non
reliable result. 


% -----------------------------------------------------------------------------------------
\subsection{Fuel cycle tools and institutions}

The originality and the efficiency of the FIT project lies in the large number
of fuel cycle tools participating in a set of exercises of a specific
functionality isolation. The Table~\ref{Tab:Code} presents participating
institution with used fuel cycle code.

\begin{table}[h]
\centering
\begin{tabular}{ |l|l| }
  \hline
  Fuel cycle code & Institution \\
  \hline
  ANICCA\cite{} & TRACTEBEL (BEL) \\
   & Univ. Católica del Maule (CHL) \\
  \hline
  CLASS\cite{Thiolliere_2018} & CNRS / IN2P3 (FRA) \\
  \hline
  CYCLUS\cite{} & Univ. of Wisconsin Madison (USA) \\
  & Univ. of Illinois (USA) \\
  \hline
  DYMOND\cite{} & Argonne National Lab (USA) \\
  \hline
  JOSSETTE\cite{} & BME (HUN) \\
  \hline
  ORION\cite{} & Oak Ridge National Lab (USA) \\
  \hline
  Tr\_Evol\cite{} & CIEMAT (ESP) \\
  \hline
  VISION\cite{jacobson2009vision} & Idaho National Lab (USA) \\
  \hline
\end{tabular}
\label{Tab:Code}
\caption{List of institutions and fuel cycle codes involved in FIT project.}
\end{table}

FIT project aims to be built on exercises related to test a specific
functionality isolation. Some tools can be used to test a functionality if they are
designed for. The participation to an exercise also depends on the availability
of participants. For those reasons, some simulators and or institution might not
be participating in some functionality isolation exercises, or might only
participate partially.
