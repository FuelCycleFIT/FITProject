% #########################################################################################
% #########################################################################################
% #########################################################################################
\section{Framework and Description Of The FIT Project}

The FIT (Functionality Isolation Test) Project is an international effort
devoted to improve the fuel cycle tools confidence. This section aims to
precisely describe and to present the framework of the project.

% -----------------------------------------------------------------------------------------
\subsection{Nuclear fuel cycle dynamic simulation tool}

Fuel cycle simulation codes~\cite{NEA2016} are developed since many years by several research and engineering institutions or consulting firms for many purposes. In the case of research and engineering institutions in charge of supporting the operated nuclear installations, fuel cycle codes are used to help and to optimize the industrial operations. For research and engineering institutions concerned by energetic transition, fuel cycle tools are used to study and analyze future trajectories for prospective reflexions on the electric component of the transition. Also, for institutions in charge of educational and training purposes in the nuclear field, this kind of tool helps understanding the physics that drive a nuclear fleet and can be used as an educational support. Consulting firms develop and use those tool to provide enlightened advices to politics. For those reasons, a fuel cycle tool can be seen as a decision making tool since results and analyses are directly or indirectly used by industrial and political worlds.

A fuel cycle simulation is based on a computer code used to model and calculate the evolution of isotopes of interest in nuclear facilities, strategic stocks and waste disposal. An important effort in tool development concerns physics model that aims to reproduce complex physics or industrial processes. A nuclear fuel cycle is then a very complex system in which isotopes evolution can be impacted by various parameters such as reactor technology deployment, fuel reprocessing strategies, etc. A fuel cycle code calculates radio-nuclides evolution in nuclear facilities from the description of a nuclear fleet. The material evolution is estimated during the irradiation process in reactors and during cooling phases is other facilities. Taking into account all the physics phenomena and industrial practices requires a huge effort in code development. For this reason, fuel cycle tools integrates many simulation simplifications.

Bias or uncertainties in fuel cycle outputs are difficult to quantify because of the multiple level of simplifications imposed by the complexity of the simulated system. If comparison with operated nuclear fleet is complex, such a process of validation is not possible for prospective study based on innovative reactors. Uncertainty or bias come from many sources. Nuclear data uncertainty has an impact on reactor calculation output used to tune fuel cycle codes~\cite{Krivtchik_2014}. Also, fuel cycle codes reactor management is usually based on simplified reactor calculations that produces a bias on neutronic data produced and used by fuel cycle simulators~\cite{Somaini_2017}. At the scale of the scenario, simplifications are also required too without always knowing precisely the precise impact of the simplification. 

Some international effort~\cite{NEA2016} propose to benchmark fuel cycle codes in order to compare output data at the scale of an operated nuclear fleet. Results produced in the framework of those works are decisive to test the ability of multiple codes to be in agreement. Nevertheless, comparison are focused on aggregated data and deviations between codes may be hard to interpret. The FIT project is built in complementarity with international benchmarks and aims to provide informations to improve confidence in data produced by fuel cycle tools.

% -----------------------------------------------------------------------------------------
\subsection{Goals of the FIT projet}

 The FIT Project was initiated in 2017 and aims to improve the
quality of data produced by fuel cycle tools. 

Increasing the ability to reproduce an operated nuclear fleet involves
increasing the complexity if the simulation tool by developing new
functionalities. We call fuel cycle code functionality a computer translation of
a physical or a technical process observed in a nuclear fleet. The
Table\ref{Tab:Funct} lists some examples of functionalities that could be
developed in replacement of a reference treatment, which is the fuel cycle code
current state.

\begin{table}[h]
\centering
\begin{tabular}{ |p{0.5\textwidth}|p{0.5\textwidth}| }
  \hline
  Reference & Functionality to develop \\
  \hline
  At each reactor loading, the reactor fresh fuel composition is constant & At each reactor loading, the reactor fresh fuel composition depends on available material isotopic composition \\
  \hline
  The reactor load factor is constant over the reactor lifetime & The reactor load factor takes into account precise industrial constraints, such as partial refueling \\
  \hline
  The mean cross sections used to perform the fuel evolution in reactor are calculated at BOC and kept constant during the cycle & The mean cross sections used to perform the fuel evolution in reactor are updated according to fuel composition \\
  \hline
  The reactor first cycles composition is not taken into account and is assumed to be the steady states composition & The exact reactor first cycles composition is used \\
  \hline
\end{tabular}
\label{Tab:Funct}
\caption{Examples of simplified and more complex functionalities.}
\end{table}

The FIT project goals is to determine which functionalities are required
according to the question that needs to be answered and the precision level to
reach. Indeed, the starting point of a fuel cycle study is a technical question
such as "In a PWR fleet in which plutonium from spent UOX fuel is reprocessed in
MOX fuel, what is the PWR-MOX fraction that perfectly balances the plutonium
produced by PWR-UOX?". The user identify then the set of output data needed to
answer the technical question. In the example above, the user needs to assess
the plutonium inventory contained in facilities between the UOX spent fuel and
the PWR MOX fuel. For illustrating, we use the first functionality of the
Table~\ref{Tab:Funct}. The FIT project would provide in this case the impact of
using a model to calculate the reactor fresh fuel composition versus using a
constant fresh fuel composition. According to this impact, the user can decide
if a new functionality devoted to the fresh fuel composition assessment is
needed or not. The user can also decide to use another tool that already include
this functionality. In addition, FIT project results could also be used to
evaluate if a functionality not present in a study could have an impact on the
observable that is used to solve the question.

The FIT project approach is very different from fuel cycle code benchmarks, as
for example the NEA benchmark~\cite{NEA2016}, since there is no code to code
comparison. The specificity here is to take advantage of the multiple tools
developed since many years with different approaches. A functionality effect is
isolated from a simple basic exercise designed for this purpose. Comparisons are
done between the exercise resolution obtained from the reference case and from
the functionality enabled. The impact of the functionality is calculated for
each participant code. This methodology provide various quantification of a
functionality impact on the fuel cycle. Some practical conclusions can be built
if there is an agreement between results.

% -----------------------------------------------------------------------------------------
\subsection{Fuel cycle tools and institutions}

The originality and the efficiency of the FIT project lies in the large number
of fuel cycle tools. The Table~\ref{Tab:Code} presents participating institution
with used fuel cycle code.

\begin{table}[h]
\centering
\begin{tabular}{ |l|l| }
  \hline
  Fuel cycle code & Institution \\
  \hline
  ANICCA\cite{} & TRACTEBEL (BEL) \\
   & Univ. Católica del Maule (CHL) \\
  \hline
  CLASS\cite{} & CNRS / IN2P3 (FRA) \\
  \hline
  CYCLUS\cite{} & Univ. of Wisconsin Madison (USA) \\
  & Univ. of Illinois (USA) \\
  \hline
  DYMOND\cite{} & Argonne National Lab (USA) \\
  \hline
  JOSSETTE\cite{} & BME (HUN) \\
  \hline
  ORION\cite{} & Oak Ridge National Lab (USA) \\
  \hline
  Tr\_Evol\cite{} & CIEMAT (ESP) \\
  \hline
  VISION\cite{} & Idaho National Lab (USA) \\
  \hline
\end{tabular}
\label{Tab:Code}
\caption{List of institutions and fuel cycle codes involved in FIT project.}
\end{table}

FIT project aims to be built on exercises related to test a specific
functionality. Some tools can be used to test a functionality if they are
designed for. The participation to an exercise also depends on the availability
of participants. For this reasons, not all the codes and institutions are
involved in an exercise.

% -----------------------------------------------------------------------------------------
\subsection{Description of the tested feature}

In the present work, we focus on the impact of using a \gls{FLM} or a \gls{FF} approach. 

A \gls{FLM} approach consists to adapt the fresh fuel composition according to
the reactor requirements and the available isotopes. A \gls{FLM} means there is a
process that connects the fresh fuel composition to the available materials,
whatever the complexity of the model. For instance, the fissile fraction is
calculated from the fissile stock quality in order to reach the required burn-up
of the reactor. A \gls{FLM} could be based on neural network, Plutonium
equivalence model, analytic functions, built-in depletion, etc. A \gls{FLM} is
usually built from physics constraints and reactor physics calculations. 

A \gls{FF} approach consists in using the same constant fissile
fraction at each fresh fuel loading whatever the isotopic vector of the
available fissile material is. Using a PWR MOX which is always loaded with a
fresh fuel that contains 7\% of plutonium regardless the $^{239}$Pu content is a
\gls{FF} approach. 

The present work aims to quantify the impact of using \gls{FLM} versus \gls{FF}
approach considering this last one as the reference. In a fuel cycle code, the
\gls{FF} approach is the easiest method to handle fresh fuel loading in a reactor. The
user simply define explicitly the fuel composition that is used in all the
simulation. Developing \gls{FLM} may be an important development process that
may require time and effort. Testing the impact of using \gls{FLM} rather than
\gls{FF} approach aims to show applications that need \gls{FLM} and study that can be
solved with \gls{FF} approach.

