% #########################################################################################
% #########################################################################################
% #########################################################################################
\section{Framework and Description Of The FIT Project}

The FIT (Functionality Isolation Test) Project is an international effort
devoted to improve the fuel cycle tools confidence. This section aims to
precisely describe and to present the framework of the project.

% -----------------------------------------------------------------------------------------
\subsection{Nuclear fuel cycle dynamic simulation tool}

Fuel cycle simulation codes~\cite{NEA2016} are developed since many years by
several research and engineering institutions or consulting firms for many
purposes. In the case of research and engineering institutions in charge of
supporting the operated nuclear installations, fuel cycle codes are used to help
and to optimize the industrial operations. For research and engineering
institutions concerned by energetic transition, fuel cycle tools are used to
study and analyze future trajectories for prospective reflexions on the electric
component of the transition. Also, for institutions in charge of educational and
training purposes in the nuclear field, this kind of tool helps understanding
the physics that drive a nuclear fleet and can be used as an educational
support. Consulting firms develop and use those tool to provide enlightened
advices to politics. For those reasons, a fuel cycle tool can be seen as a
decision making tool since results and analyses are directly or indirectly used
by industrial and political worlds.

A fuel cycle simulation is based on a computer code used to model and calculate
the evolution of isotopes of interest in nuclear facilities, strategic stocks
and waste disposal. An important effort in tool development concerns physics
model that aims to reproduce complex physics or industrial processes. A nuclear
fuel cycle is then a very complex system in which isotopes evolution can be
impacted by various parameters such as reactor technology deployment, fuel
reprocessing strategies, etc. A fuel cycle code calculates radio-nuclides
evolution in nuclear facilities from the description of a nuclear fleet. The
material evolution is estimated during the irradiation process in reactors and
during cooling phases is other facilities. Taking into account all the physics
phenomena and industrial practices requires a huge effort in code development.
For this reason, fuel cycle tools integrates many simulation simplifications.

Bias or uncertainties in fuel cycle outputs are difficult to quantify because of
the multiple level of simplifications imposed by the complexity of the simulated
system. If comparison with operated nuclear fleet is complex, such a process of
validation is not possible for prospective study based on innovative reactors.
Uncertainty or bias come from many sources. Nuclear data uncertainty has an
impact on reactor calculation output used to tune fuel cycle
codes~\cite{Krivtchik_2014}. Also, fuel cycle codes reactor management is
usually based on simplified reactor calculations that produces a bias on
neutronic data produced and used by fuel cycle simulators~\cite{Somaini_2017}.
At the scale of the scenario, simplifications are also required too without
always knowing precisely the precise impact of the simplification. 

Some international effort~\cite{NEA2016} propose to benchmark fuel cycle codes
in order to compare output data at the scale of an operated nuclear fleet.
Results produced in the framework of those works are decisive to test the
ability of multiple codes to be in agreement. Nevertheless, comparison are
focused on aggregated data and deviations between codes may be hard to
interpret. The FIT project is built in complementarity with international
benchmarks and aims to provide informations to improve confidence in data
produced by fuel cycle tools.

% -----------------------------------------------------------------------------------------
\subsection{Goals and intended impact of the FIT project}

The FIT Project was initiated in 2017 and aims to improve the confidence into
data produced by fuel cycle tools. The first goal is to animate a community of
fuel cycle specialists focused on the question of the confidence in outputs. The
second goal is to determine the minimum level of details a fuel cycle simulator
should have according to the type of study and the required confidence level.
The project relies on the wide variety of fuel cycle simulators with a wide
range of complexity level are developed

Increasing the ability to reproduce an operated nuclear fleet involves
increasing the complexity of the simulation tool by developing new
functionalities. A fuel cycle code functionality is the translation into
computer software language of a physical or technical process related to nuclear
facilities. The Table\ref{Tab:Funct} lists some examples of functionalities that
could be developed in replacement of a reference treatment represented by the
fuel cycle code current state.

\begin{table}[h]
\centering
\begin{tabular}{ |p{0.5\textwidth}|p{0.5\textwidth}| }
  \hline
  Reference & Functionality to develop \\
  \hline
  At each reactor loading, the reactor fresh fuel composition is constant & At
  each reactor loading, the reactor fresh fuel composition depends on available
  material isotopic composition \\
  \hline
  The reactor load factor is constant over the reactor lifetime & The reactor
  load factor takes into account precise industrial constraints, such as partial
  refueling \\
  \hline
  The mean cross sections used to perform the fuel evolution in reactor are
  calculated at BOC and kept constant during the cycle & The mean cross sections
  used to perform the fuel evolution in reactor are updated according to fuel
  composition \\
  \hline
  The reactor first cycles composition is not taken into account and is assumed
  to be the steady states composition & The exact reactor first cycles
  composition is used \\
  \hline
\end{tabular}
\label{Tab:Funct}
\caption{Examples of simplified and more complex functionalities.}
\end{table}

The FIT project approach consists of isolating a functionality effect from
simple basic exercises designed for this purpose. Each exercise is focused on one type of output that is linked to a type of problem covered by the fuel cycle study. For instance, the total mass of plutonium or the isotopic composition of plutonium will be used as output of interest if fuel cycle studies deal with the recycling of nuclear fuels. Minor actinides production could be added to the analyses to take into account fuel cycle studies concerned by radio-protection. Once the choice of output of interest is made, the effect of the functionality is quantified by specific estimators that are computed with the functionality enabled and the reference case. Each participant propose a resolution for the exercise and some conclusions can be built according to the level of agreement of participants.

With this methodology, the FIT project could give informations on which
functionalities are required according to the question that needs to be answered
and the precision level to reach. A fuel cycle code user starts from a technical
question, such as "In a PWR fleet in which plutonium from spent UOX fuel is
reprocessed in MOX fuel, what is the PWR-MOX fraction that perfectly balances
the plutonium produced by PWR-UOX?". The user identify then the set of output
data needed to answer the technical question. In the example above, the user
needs to assess the plutonium inventory contained in facilities between the UOX
spent fuel and the PWR MOX fuel. The user can then use the FIT project results
to decide what are the code required functionalities or which functionalities
will produce non reliable result. 

% -----------------------------------------------------------------------------------------
\subsection{Fuel cycle tools and institutions}

The originality and the efficiency of the FIT project lies in the large number
of fuel cycle tools. The Table~\ref{Tab:Code} presents participating institution
with used fuel cycle code.

\begin{table}[h]
\centering
\begin{tabular}{ |l|l| }
  \hline
  Fuel cycle code & Institution \\
  \hline
  ANICCA\cite{} & TRACTEBEL (BEL) \\
   & Univ. Católica del Maule (CHL) \\
  \hline
  CLASS\cite{Thiolliere_2018} & CNRS / IN2P3 (FRA) \\
  \hline
  CYCLUS\cite{} & Univ. of Wisconsin Madison (USA) \\
  & Univ. of Illinois (USA) \\
  \hline
  DYMOND\cite{} & Argonne National Lab (USA) \\
  \hline
  JOSSETTE\cite{halasz2018development} & BME (HUN) \\
  \hline
  ORION\cite{Feng_ORION_2016} & Oak Ridge National Lab (USA) \\
  \hline
  TR\_EVOL\cite{Alvarez-Velarde2010} & CIEMAT (ESP) \\
  \hline
  VISION\cite{jacobson2009vision} & Idaho National Lab (USA) \\
  \hline
\end{tabular}
\label{Tab:Code}
\caption{List of institutions and fuel cycle codes involved in FIT project.}
\end{table}

FIT project aims to be built on exercises related to test a specific
functionality. Some tools can be used to test a functionality if they are
designed for. The participation to an exercise also depends on the availability
of participants. For those reasons, not all the codes and institutions are
involved in an exercise.
