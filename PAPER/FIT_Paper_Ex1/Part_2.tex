% #########################################################################################
% #########################################################################################
% #########################################################################################
\section{Presentation Of The FIT Project}

The FIT (Functionality Isolation Test) Project is an international effort devoted to improve the fuel cycle tools confidence. This section aims to precisely describe the project.


% -----------------------------------------------------------------------------------------
\subsection{Goals of the projet}

A nuclear fuel cycle is a very complex system in which isotopes evolution can be impacted by various parameters such as reactor technology deployment, fuel reprocessing strategies, etc. A fuel cycle code calculates radio-nuclides evolution in nuclear facilities from the description of a nuclear fleet. The material evolution is estimated during the irradiation process in reactors and during cooling phases is other facilities. Taking into account all the physics phenomena and industrial practices requires a huge effort in code development. For this reason, fuel cycle tools integrates many simulation simplifications. A lot of fuel cycle tools are developed and used worldwide~\cite{COSI6_2015, VanDenDurpel2015, Huff2016} with large level of complexity or range of applications. The FIT Project was initiated in 2017 and aims to improve the quality of data produced by fuel cycle tools. 

Increasing the ability to reproduce an operated nuclear fleet involves increasing the complexity if the simulation tool by developing new functionalities. We call fuel cycle code functionality a computer translation of a physical or a technical process observed in a genuine nuclear fleet. Developing a new functionality is time consuming and an arbitration has to be done between the code precision and development time. In this framework, the FIT project aims to determine which functionalities are required according to the question that needs to be answered. Indeed, some scientific questions may be answered from output data that requires a minimum set of functionality to be properly treated. On the other hand, some questions could be answered from a code based on a high level a simplifications. 

Test Features
    - Not a benchmark, pas de comparaison code a code
    - Simple exercices
    - Comparaison entre feature VS pas la feature
    - Check si les CCL sont similaires. 
    
Institutions and Participants
    - All + Code + list
    
Functionality tested in this work
    - Definition FLM
    - Definition FF







% -----------------------------------------------------------------------------------------
\subsection{Fuel cycle tools and institutions}



% -----------------------------------------------------------------------------------------
\subsection{Description of the tested feature}


For instance, treating a reactor ... mixed fuel?

