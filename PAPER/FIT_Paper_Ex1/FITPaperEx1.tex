\documentclass[review]{elsarticle}

\usepackage{lineno,hyperref}
\modulolinenumbers[5]

\journal{To Discuss With FIT Collaboration}

%%%%%%%%%%%%%%%%%%%%%%%
%% Elsevier bibliography styles
%%%%%%%%%%%%%%%%%%%%%%%
%% To change the style, put a % in front of the second line of the current style and
%% remove the % from the second line of the style you would like to use.
%%%%%%%%%%%%%%%%%%%%%%%

%% Numbered
%\bibliographystyle{model1-num-names}

%% Numbered without titles
%\bibliographystyle{model1a-num-names}

%% Harvard
%\bibliographystyle{model2-names.bst}\biboptions{authoryear}

%% Vancouver numbered
%\usepackage{numcompress}\bibliographystyle{model3-num-names}

%% Vancouver name/year
%\usepackage{numcompress}\bibliographystyle{model4-names}\biboptions{authoryear}

%% APA style
%\bibliographystyle{model5-names}\biboptions{authoryear}

%% AMA style
%\usepackage{numcompress}\bibliographystyle{model6-num-names}

%% `Elsevier LaTeX' style
\bibliographystyle{elsarticle-num}
%%%%%%%%%%%%%%%%%%%%%%%

\begin{document}

\begin{frontmatter}

\title{Analysis Of Fuel Loading Management in Fuel Cycle Simulation Tools In The Framework Of The FIT Project}
%\tnotetext[mytitlenote]{Fully documented templates are available in the elsarticle package on \href{http://www.ctan.org/tex-archive/macros/latex/contrib/elsarticle}{CTAN}.}

%% Group authors per affiliation. 
% Rules for Author List : Two groups 
% Group 1 = Main contributors (calculations and/or significative helping writing)
% Group 2 = Other membre of the FIT Project
% Main writer is Corresponding Autor

% Group 1

\author[IPNO]{X.~Doligez}
\author[ANL]{B.~Feng}
\author[BUD]{M.~Halasz}
\author[SCK]{A.~Hernandez}
\author[MAULE]{I.~Merino}
\author[MAD]{B.~Mouginot}
\author[CIEMAT]{A.V.~Skarbeli}

\author[SUB]{N. Thiolli\`ere \corref{cor1}}
\ead{nicolas.thiolliere@subatech.in2p3.fr}

% Group 2
\author[CIEMAT]{F.~Alvarez-Velarde}
\author[ORNL]{E.~E.~Davidson}
\author[TRACT]{H.~Druenne}
\author[IPNO]{M.~Ernoult}
\author[USC]{R.~Flanagan}
\author[INL]{R.~Hays}
\author[UI]{K.~Huff}
\author[BUD]{M.~Szieberth}
\author[TRACT]{B.~Vermeeren}
\author[CIEMAT]{A.~Villacorta}
\author[MAD]{P.~Wilson}

\cortext[cor1]{Corresponding author}

\address[IPNO]{Institut de Physique Nucléaire d’Orsay, CNRS-IN2P3/Univ, Paris-Sud, France}
\address[ANL]{Argonne National Laboratory, 9700 Cass Ave., Lemont, IL 60439, USA}
\address[BUD]{Budapest University of Technology and Economics (BME), Institute of Nuclear Techniques, 1111 Budapest, Müegyetem rkp. 3-9, Hungary}
\address[SCK]{Studiecentrum voor kernenergie - Centre d'étude de l'énergie nucléaire (SCK-CEN), Boeretang 200, Mol, Belgium}
\address[MAULE]{Catholic University of the Maule, Av. San Miguel 3605, Talca, Chile}
\address[MAD]{Univ. of Wisconsin Madison, Department of Nuclear Engineering and Engineering Physics, Madison, WI, United States}
\address[SUB]{Subatech, IMTA-IN2P3/CNRS-Universit\'e, Nantes, F-44307, France}
\address[ORNL]{Oak Ridge National Laboratory, Building 5700, Mail Stop 6172, Oak Ridge, TN 37831, United States}
\address[TRACT]{Tractebel Engie, Boulevard Simón Bolívar 34-36, 1000 Brussels, Belgium}
\address[USC]{University of South Carolina, Nuclear Engineering Program, Columbia, SC 29201, United States}
\address[INL]{Idaho National Laboratory, 2525 Fremont Ave., Idaho Falls, ID 83402, USA}
\address[UI]{University of Illinois, Department of Nuclear, Plasma, and Radiological Engineering, United States}
\address[CIEMAT]{CIEMAT, Avda. Complutense, 40, 28040 Madrid, Spain}

\begin{abstract}
Abstract beginning...


In this paper, the first tested functionality is presented. The impact of the fuel composition dependency with stock versus a fixed fraction approach is tested. Results from different methodologies are compared.
\end{abstract}

\begin{keyword}
Nuclear Fuel Cycle \sep Simulation \sep Functionality \sep Benchmark
\end{keyword}

\end{frontmatter}

\linenumbers

% #########################################################################################
% #########################################################################################
% #########################################################################################
\section{Introduction}

% #########################################################################################
% #########################################################################################
% #########################################################################################
\section{Presentation Of The FIT Project}

The FIT (Functionality Isolation Test) Project is an international effort devoted to improve the fuel cycle tools confidence. This section aims to precisely describe the project.


% -----------------------------------------------------------------------------------------
\subsection{Goals of the projet}

A nuclear fuel cycle is a very complex system in which isotopes evolution can be impacted by various parameters such as reactor technology deployment, fuel reprocessing strategies, etc. A fuel cycle code calculates radio-nuclides evolution in nuclear facilities from the description of a nuclear fleet. The material evolution is estimated during the irradiation process in reactors and during cooling phases is other facilities. Taking into account all the physics phenomena and industrial practices requires a huge effort in code development. For this reason, fuel cycle tools integrates many simulation simplifications. A lot of fuel cycle tools are developed and used worldwide~\cite{COSI6_2015, VanDenDurpel2015, Huff2016} with large level of complexity or range of applications. The FIT Project was initiated in 2017 and aims to improve the quality of data produced by fuel cycle tools. 

Increasing the ability to reproduce an operated nuclear fleet involves increasing the complexity if the simulation tool by developing new functionalities. We call fuel cycle code functionality a computer translation of a physical or a technical process observed in a genuine nuclear fleet. Developing a new functionality is time consuming and an arbitration has to be done between the code precision and development time. In this framework, the FIT project aims to determine which functionalities are required according to the question that needs to be answered. Indeed, some scientific questions may be answered from output data that requires a minimum set of functionality to be properly treated. On the other hand, some questions could be answered from a code based on a high level a simplifications. 

Test Features
	- Not a benchmark, pas de comparaison code a code
	- Simple exercices
	- Comparaison entre feature VS pas la feature
	- Check si les CCL sont similaires. 
	
Institutions and Participants
	- All + Code + list
	
Functionality tested in this work
	- Definition FLM
	- Definition FF







% -----------------------------------------------------------------------------------------
\subsection{Fuel cycle tools and institutions}



% -----------------------------------------------------------------------------------------
\subsection{Description of the tested feature}


For instance, treating a reactor ... mixed fuel?

% #########################################################################################
% #########################################################################################
% #########################################################################################
\section{Exercice definition}



% -----------------------------------------------------------------------------------------
\subsection{Specification}

% -----------------------------------------------------------------------------------------
\subsection{Methodology}

% -----------------------------------------------------------------------------------------
\subsection{Output observables}






% #########################################################################################
% #########################################################################################
% #########################################################################################
\section*{References}

\bibliography{mybibfile}

\end{document}
