% #########################################################################################
% #########################################################################################
% #########################################################################################
\section{Exercise design}

% -----------------------------------------------------------------------------------------
\subsection{Description of the tested functionality}

The present work focuses on the impact of using a \gls{FLM} or a \gls{FF}
approach. 

A \gls{FLM} approach consists to adapt the fresh fuel composition according to
the reactor requirements and the available material isotopic compositions. A
\gls{FLM} connects the fresh fuel composition to the available materials,
regardless to the complexity of the model. For instance, the fissile fraction is
calculated from the fissile stock quality in order to reach the required burn-up
of the reactor. A \gls{FLM} could be based on neural network, Plutonium
equivalence model, analytic functions, built-in depletion, etc. A \gls{FLM} is
usually built from physics constraints and reactor physics calculations. 

A \gls{FF} model consists in using the same constant fissile fraction at each
fresh fuel loading regardless of the isotopic composition of the available
fissile material. Using a \gls{PWR} \gls{MOX} which is always loaded with a
fresh fuel that contains 7\% of plutonium regardless the $^{239}$Pu content is a
\gls{FF} approach. 

The present work aims to quantify the impact of using \gls{FLM} versus \gls{FF}
approach considering this later as the reference, as it should have more physics
refinement. In a fuel cycle simulator, the \gls{FF} approach is the cheapest
method to deal with fresh reprocessed fuel fabrication. Developing \gls{FLM}
may be an important development process requiring time and effort.
Testing the impact of using \gls{FLM} rather than \gls{FF} aims to identify the
study that require \gls{FLM} and the one solvable with \gls{FF}.

\subsection{Output of interest}

Fuel fabrication models can impact fuel cycle simulation in many different ways.
This exercise aims to understand some of their impacts on an overall fuel cycle
calculation. Change in the plutonium content in a \gls{MOX} fuel could impacts
the fuel cycle in 3 majors ways:

\begin{itemize}
    \item the global amount of plutonium in the simulation, variation in the
        plutonium content in the \gls{MOX} fuel, might lead in a increase of the
        amount of plutonium burnt in the \gls{PWR} reactor, or a change in the
        breeding/burning ratio in a \gls{SFR} reactor, impacting the overall amount of
        plutonium in the simulation.
    \item the location of the plutonium, indeed the amount of plutonium in the
        \gls{MOX} fuel will shift the localization of the plutonium form the front-end
        the back-end of the cycle, which could change its availability for
        other use (deployment of new reactor, fabrication of other fuels\ldots),
    \item the variation of the global amount of plutonium relative to the amount
        loaded in the fuel, the breeding/burning ratio of the plutonium will
        change the amount of plutonium in reactor.
\end{itemize}

This might not be exhaustive list of fuel fabrication models impact on a fuel
cycle, but this work will only focus on them and will not consider potential
consequence on the fuel composition change after depletion due to
over/under-estimate the amount of plutonium in the \gls{MOX} fuel.

In order to investigate those impacts across multiple fuel cycle simulation tool
and associated models, the following experience has been designed.

% -----------------------------------------------------------------------------------------
\subsection{Exercise specifications}

The exercise is divided in two independent parts related to the reactor
involved. \gls{PWR} and \gls{SFR} will be considered in this work.

The schematic representation of the fuel cycle is shown on
Figure~\ref{fig:FuelCycle}. The fuel cycle includes a fissile stock that
contains enough plutonium to operate one reactor cycle and a fertile stock
filled of depleted uranium. A fresh fuel fabrication plant that uses the heavy
elements of the stocks build the fresh fuel of the reactor (PWR or \gls{SFR})
according the its technical requirements. The reactor spent fuel is sent to a
dedicated stock.

\begin{figure}[h]
    \begin{center}
        \includegraphics[width = 0.99\textwidth]{FIG/FuelCycleDiagram.pdf}
        \caption{Representation of the simulated fuel cycle facilities.}
        \label{fig:FuelCycle}
    \end{center}
\end{figure}

The time frame of the simulation corresponds to the reactor cycle. From the
relation between the fuel cycle time $\Delta t$, the reactor thermal power
$P_{th}$, the heavy nuclide mass $M$ and the reactor burn-up $BU$, we have : 

\begin{equation}
    \Delta t = \frac{BU \; M}{P_{th}}
\end{equation}

At t = 0, the fabrication plant builds the fresh fuel according to reactor
requirements. To avoid plutonium isotopes decay, the fabrication time is zero
and the reactor is thus loaded instantaneously. A complete fuel cycle is run and
the spent fuel is sent to stock when the required BU is reached.

Concerning the plutonium vector used to build the \gls{MOX} fuel, a large range has
been imposed for plutonium isotopes. Indeed, the impact of a \gls{FLM} in
relation to \gls{FF} approach on a fuel cycle calculation will increase with the
plutonium fraction deviation from a reference case. To take an example, let
consider a small fissile content plutonium. A \gls{FF} approach would for
instance load the fuel at 6\% of plutonium in the fresh fuel since a \gls{FLM}
approach would give 12\%. On the other hand, if the plutonium vector used to
build the fuel is very close from the reference plutonium used to design the
\gls{FF} approach, \gls{FF} and \gls{FLM} would produce similar results.
Perfectly well tuned scenario for which plutonium stocks are well known and
approximately constant with time can be properly solved with a \gls{FF}
approach. 

A wide range of scenario applications is considered here. For instance, the
plutonium vector produced from a small burn-up discharge CANDU is characterized
by a high fissile fraction~\cite{Guillemin_2010}. The multi-recycling of the
plutonium into \gls{PWR} may produce a high even isotopes fraction in the plutonium
vector~\cite{Courtin_2016}. Very specific plutonium isotopic fraction can be
seen from scenarios involving \gls{SFR} reactors. Also, industrial parameters, such
has spent fuel cooling time, may have a strong impact on $^{241}$Pu fraction.
For those reasons, the plutonium isotopic range has been chosen very large to be
representative of a wide range of applications although being realistic enough.
The Table~\ref{Tab:PuVector} present minimum and maximum isotopic fraction of
plutonium isotopes used in the framework of this work.

\begin{table}[h]
\centering
\begin{tabular}{ |l|l|l| }
  \hline
  Isotope & Min. Mass Fr. (wgt. \%) & Max. Mass Fr. (wgt. \%) \\
  \hline
  Pu-238/TRU & 0  & 10 \\
  \hline
  Pu-239/TRU & 25 & 90 \\
  \hline
  Pu-240/TRU & 10 & 40 \\
  \hline
  Pu-241/TRU & 0  & 25 \\
  \hline
  Pu-242/TRU & 0  & 30 \\
  \hline
  Am-241/TRU & 0  & 10 \\
  \hline
\end{tabular}
\label{Tab:PuVector}
\caption{Minimum and maximum mass fraction in weight \% for plutonium vector at
        reactor beginning of cycle used in the framework of this work}
\end{table}

% -----------------------------------------------------------------------------------------
\subsection{Problem-solving methodology}

In order to investigate the influence of using a \gls{FLM} rather than a
\gls{FF} approach on fuel cycle simulations, the following experiment have been
designed. Following the FIT project spirit, no code to code comparison will be
done. For each simulator used to solve this problem, fuel models will be
compared within the same simulator, allowing to evaluate difference between two
almost identical simulation: same simulator, reactor description, depletion
algorithm, \ldots the only difference being the method used to build the fresh fuel.

The experiment consists in running a small fuel cycle calculation according to
specifications defined above. We use a method that we call "Wide Parametric
Sweeping" method. The principle of this method is to fill from a random 
sampling method the plutonium isotopic space defined in the Table~\ref{
Tab:PuVector}. The global set of plutonium isotopic vectors that will be run 
is called the design of experiment. Each plutonium isotopic vector will be run 
twice. One using a \gls{FLM} approach for which the plutonium fraction at 
reactor \gls{BOC} directly depends on the plutonium isotopic composition. The 
second run uses \gls{FF} approach for which the plutonium fraction at \gls{BOC}
is always the same. 

Each code produces for each reactor, \gls{PWR} and \gls{SFR}, two sets of output
data representing \gls{FLM} and \gls{FF} runs. For each case, the output of
interest are the mass of plutonium at \gls{BOC} $M_{Pu}^{BOC}$ and \gls{EOC}
$M_{Pu}^{BOC}$ and the plutonium fraction $\lambda_{\mathrm{Pu}}^{BOC}$ at
\gls{BOC}. Future works will be dedicated to extend the range of output, up to
plutonium isotopic composition or minor actinides production. 

In order to measure the influence of the use of \gls{FF} versus \gls{FLM} on the
global amount of plutonium in the simulation, the location of the plutonium and
the speed of variation of the plutonium inventory, three estimators have been
defined.


\subsection{Estimators\label{subsec:estimator}}

From output data of interest, estimators are built to compare simulations run
from \gls{FF} and \gls{FLM}. Inside a fuel cycle, two effects are dissociated.
The first effect concerns global outputs which is a data estimated on the entire
fuel cycle. This could be the total inventory of plutonium or minor actinides
for for instance. The second effect concerns local outputs and is associated to
the inventory in a specific facility or disposal unit. This could be the
plutonium mass in a strategic stock. A measured global effect necessarily
induces a local effect. Indeed, if the total plutonium mass at a specific time
calculated from \gls{FLM} approach is higher than the \gls{FF} approach, that
means that one or more facilities are affected. On the contrary, a local effect
does not necessarily mean there is a global effect. A deviation in plutonium
mass in a specific stock can be compensated in another facility.


\subsubsection{Estimator 1}

The first estimator aims to measure the difference on the plutonium enrichment
in the \gls{MOX} fuel between the \gls{FLM} and \gls{FF}. This estimator directly
impacts the amount of plutonium present in the back-end part of the fuel cycle
and is then a local estimator. The estimator 1 has been defined as:

\begin{equation}
    \delta{\lambda}(i) =
        \frac{\left(\lambda_{\mathrm{Pu}}^{BOC}(i)\right)_{FML}
              - \left(\lambda_{\mathrm{Pu}}^{BOC}(i)\right)_{FF}}
              {\left(\lambda_{\mathrm{Pu}}^{BOC}(i)\right)_{FF}},
\end{equation}

where $\lambda_i$ represents the fraction of plutonium at \gls{BOC} or at
\gls{EOC} for the plutonium composition $i$ for the \gls{FLM} or the \gls{FF}.
The higher the estimator 1 is, the higher the plutonium fraction at \gls{BOC} is
for the \gls{FLM} compared to \gls{FF}. This estimator is used in the case of
\gls{PWR} and \gls{SFR} analysis.

\subsubsection{Estimator 2}

The second estimator aims to estimate the relative speed of plutonium
consumption in the reactor between the \gls{FLM} and the \gls{FF} approach. The
estimator 2 is defined as:

\begin{equation}
    \delta{\frac{\Delta M}{M}}(i) =
        \frac{\left(\frac{\Delta M}{M}(i)\right)_{FML}
              - \left(\frac{\Delta M}{M}(i)\right)_{FF}}
             {\left(\frac{\Delta M}{M}(i)\right)_{FF}},
\end{equation}
where $\frac{\Delta M}{M}_{i}$ is defined as:
\begin{equation}
    \frac{\Delta M}{M}(i) = \frac{M_{Pu}^{BOC}(i) -
    M_{Pu}^{EOC}(i)}{M_{Pu}^{BOC}(i)}
\end{equation}

Estimator 2 represents a global effect since the plutonium consumption speed has
an impact on total plutonium mass. This estimator is suitable for reactor
simulations characterized by plutonium decrease and is used for \gls{PWR} analysis.

\subsubsection{Estimator 2b}

The estimator 2 can tend toward infinity if reactor is iso-breeder. To avoid
such a behavior, an estimator 2b has been defined as following : 

\begin{equation}
    \delta \frac{\Delta M}{M}(Pu_i) = \frac{\Delta M}{M}(Pu_i)_{FLM} - \frac{\Delta M}{M}(Pu_i)_{FF}
\end{equation}

Estimator 2b measures a global effect and will be used for \gls{SFR} results.

\subsubsection{Estimator 3}

The third estimator shows the absolute speed of plutonium consumption in the
reactor between the \gls{FLM} and the \gls{FF} approach and is defined as
following :

\begin{equation}
    \delta{\frac{\Delta M}{T}}(i) =
        \frac{\left(\frac{\Delta M}{T}(i)\right)_{FML}
              - \left(\frac{\Delta M}{T}(i)\right)_{FF}}
             {\left(\frac{\Delta M}{T}(i)\right)_{FF}},
\end{equation}
where $\frac{\Delta M}{T}_{i}$ is defined as:
\begin{equation}
    \frac{\Delta M}{T}(i) = \frac{M_{Pu}^{BOC}(i) -
    M_{Pu}^{EOC}(i)}{T}
\end{equation}

Estimator 3 characterizes a global effect and is used in the case of \gls{PWR}
simulations analysis.

