% #########################################################################################
% #########################################################################################
% #########################################################################################
\section{Exercise design}

% -----------------------------------------------------------------------------------------
\subsection{Description of the tested functionality}

In the present work, we focus on the impact of using a \gls{FLM} or a \gls{FF} approach. 

A \gls{FLM} approach consists to adapt the fresh fuel composition according to
the reactor requirements and the available isotopes. A \gls{FLM} means there is a
process that connects the fresh fuel composition to the available materials,
whatever the complexity of the model. For instance, the fissile fraction is
calculated from the fissile stock quality in order to reach the required burn-up
of the reactor. A \gls{FLM} could be based on neural network, Plutonium
equivalence model, analytic functions, built-in depletion, etc. A \gls{FLM} is
usually built from physics constraints and reactor physics calculations. 

A \gls{FF} approach consists in using the same constant fissile
fraction at each fresh fuel loading whatever the isotopic vector of the
available fissile material is. Using a PWR MOX which is always loaded with a
fresh fuel that contains 7\% of plutonium regardless the $^{239}$Pu content is a
\gls{FF} approach. 

The present work aims to quantify the impact of using \gls{FLM} versus \gls{FF}
approach considering this last one as the reference. In a fuel cycle code, the
\gls{FF} approach is the easiest method to handle fresh fuel loading in a reactor. The
user simply define explicitly the fuel composition that is used in all the
simulation. Developing \gls{FLM} may be an important development process that
may require time and effort. Testing the impact of using \gls{FLM} rather than
\gls{FF} approach aims to show applications that need \gls{FLM} and study that can be
solved with \gls{FF} approach.

\subsection{Output of interest}

Fuel fabrication models can impacts fuel cycle simulation in different ways.
This exercise aims to understand some of their impacts on an overall fuel cycle
calculation. Change in the plutonium content in a newly built MOX fuel will
impacts the fuel cycle in 3 majors ways:

\begin{itemize}
    \item the global amount of plutonium in the simulation, variation in the
        plutonium content in the MOX fuel, might lead in a increase of the
        amount of plutonium burnt in the PWR reactor, or a change in the
        breeding/burning ratio in a SFR reactor, impacting the overall amount of
        plutonium in the simulation.
    \item the location of the plutonium, indeed the amount of plutonium in the
        MOX fuel will shift the localisation of the plutonium form the front-end
        the back-end of the cycle, which could change its availability for
        other use (deployment of new reactor, fabrication of other fuels\ldots),
    \item the variation of the global amount of plutonium relative to the amount
        loaded in the fuel, the breeding/burning ratio of the plutonium will
        change the amount of plutonium in reactor.
\end{itemize}

This might not be exhaustive list of fuel fabrication models impact on a fuel
cycle, but this work will only focus on them and will not consider potential
consequence on the fuel composition change after depletion due to
over/under-estimate the amount of plutonium in the MOX fuel.

In order to investigate those impacts across multiple fuel cycle simulation tool
and associated models, the following experience has been designed.

% -----------------------------------------------------------------------------------------
\subsection{Specification}

% -----------------------------------------------------------------------------------------
\subsection{Method}
In order to investigate the influence of fuel fabrication models on fuel cycle
simulations, the following experiment have been designed. Following the FIT
project spirit, no code to code comparison will be done. For each simulator used
to solve this problem, fuel models will be compared within the same simulator,
allowing to evaluate difference between two almost identical simulation: same
simulator, reactor description, depletion algorithm, \ldots the only difference
being the fuel fabrication model used.

The experiment consists in running a small fuel cycle, composed by a infinite
plutonium source, a fuel fabrication and a reactor, for multiple plutonium
isotopic compositions (about 1000). Each plutonium isotopic will be run twice,
one using a fuel fabrication model using a fixed fraction, always the same
defined in the experiment specification, and a second one using an advanced fuel
loading model (based on plutonium equivalent theory\ref{} or pre-trained neural
networks).

The different plutonium composition are sampled uniformly on a predefined
user-space (that may vary depending of the simulator used). Moreover, even if
it is understood that an largely under/over-estimated plutonium content in the
MOX fuel, may compromise the reactor well operation, it is assumed in the
following that all fuel loading models used (fixed fraction or more complex fuel
loaded models), will produced proper MOX fuels.

In order to measure the influence of the used of fixed fraction versus a fuel
loading models on the global amount of plutonium in the simulation, the location
of the plutonium and the speed of variation of the plutonium inventory, three
estimators have been defined.

\subsection{Estimators}
\subsubsection{Estimator 1}
The first estimator aims to the difference on the plutonium enrichment in the MOX
fuel between the \gls{FLM} and \gls{FF} which directly impacts the amount of
plutonium present in the back-end part of the fuel cycle. The estimator 1 has
been defined as:
\begin{equation}
    \delta{\lambda}(i) =
        \frac{\left(\lambda_{\mathrm{Pu}}^{BOC}(i)\right)_{FML}
              - \left(\lambda_{\mathrm{Pu}}^{BOC}(i)\right)_{FF}}
              {\left(\lambda_{\mathrm{Pu}}^{BOC}(i)\right)_{FF}},
\end{equation}

where $\lambda_i$ represents the fraction of plutonium at \gls{BOC} or at
\gls{EOC} for the plutonium composition $i$ for the \gls{FLM} or the \gls{FF}.

\subsubsection{Estimator 2}
The second estimator aims to relative speed of plutonium amount evolution in the
reactor between the \gls{FLM} and the \gls{FF}. The estimator 2 is defined as:
\begin{equation}
    \delta{\frac{\Delta M}{M}}(i) =
        \frac{\left(\frac{\Delta M}{M}(i)\right)_{FML}
              - \left(\frac{\Delta M}{M}(i)\right)_{FF}}
             {\left(\frac{\Delta M}{M}(i)\right)_{FF}},
\end{equation}
where $\frac{\Delta M}{M}_{i}$ is defined as:
\begin{equation}
    \frac{\Delta M}{M}(i) = \frac{M_{Pu}^{BOC}(i) -
    M_{Pu}^{EOC}(i)}{M_{Pu}^{BOC}(i)}
\end{equation}


\subsubsection{Estimator 3}


\begin{equation}
    \delta{\frac{\Delta M}{T}}(i) =
        \frac{\left(\frac{\Delta M}{T}(i)\right)_{FML}
              - \left(\frac{\Delta M}{T}(i)\right)_{FF}}
             {\left(\frac{\Delta M}{T}(i)\right)_{FF}},
\end{equation}
where $\frac{\Delta M}{T}_{i}$ is defined as:
\begin{equation}
    \frac{\Delta M}{T}(i) = \frac{M_{Pu}^{BOC}(i) -
    M_{Pu}^{EOC}(i)}{T}
\end{equation}

% -----------------------------------------------------------------------------------------
\subsection{Output observables}


