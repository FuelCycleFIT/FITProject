\section{Conclusions and Perspectives}
\label{sec:conclusion}

Substantial resources have been invested in a variety of Nuclear Fuel Cycle
tools, whether to serve specific purposes, or as they expanded to more general
purpose, but confidence in each new analysis remains fragile because of a lack
of generic ability to develop confidence in such tools.
Fuel cycle simulators can be used to answer different questions that requires
different level of precision. To increase the confidence level of studies, all
institutions are willing to increase the level of complexity of the software
even if it does not necessary improve the precision of the calculations. 

This paper presents the FIT project which aims to provide a rigorous scientific
tools allowing researchers/developers/analysts to determine whether or not a
feature, a functionality, a model refinement is required to perform a nuclear
fuel cycle study within a certain precision goal.

The FIT project approach consists in testing code functionality by an intra-code
comparison of calculation with and without the feature. The FIT project is not a
benchmark and there is no inter-simulator comparison but aims to perform
intra-simulator comparison, comparing the simulator output obtained with a single
simulator, with and without the feature/functionality to test. The strength of
this project relies on the diversity of simulator used to solve the different
problems allowing to reach nearly simulator-agnostic conclusion. Following this
principle, the FIT project will provide informations about which functionalities
are required according to the question that needs to be answered and the
precision level to reach.

The work presented here is dedicated fuel cycle simulators ability to vary fresh
fuel compositions according to the available materials. Two treatments have been
considered: a fixed enrichment for fresh fuel, regardless of the isotopic
composition, and a model to adjust the fresh fuel composition in accordance
to the isotopic composition. In this first exercise, the output estimators focus
on availability of the plutonium in the fuel cycle, so will probably be more
valuable for fuel cycle study involving plutonium recycling.

In agreement with the FIT philosophy, all calculations were performed with a
fixed fraction and with the use of a model that calculate fissile enrichment for
fresh fuels. Inter-software comparison have been done but each code is tested
against itself, \gls{FF} vs \gls{FLM} on 2 different reactors, across 5 different
simulators, (CLASS, ANICCA, TR\_EVOL and CYCLUS) for the \gls{PWR} and (CLASS,
TR\_EVOL, JOSETTE) for the \gls{SFR}. The importance of adjusting fresh fuel
compositions have been tested around 3 main questions : 
\begin{itemize}
    \item the global plutonium inventory estimation and its evolution rate,
    \item the local plutonium inventories.
\end{itemize}

Results shows that the \gls{FF} calculations leads to difference between
\gls{FF} and the \gls{FLM} calculation between $21\%$ and $40\%$ for the fresh
\gls{PWR}-\gls{MOX} fuel plutonium enrichment. Differences for \gls{SFR}
calculation are observed between $9\%$ and $13\%$.  The \gls{BOC} plutonium
enrichment is directly related to the amount of spent fuel reprocessing fuel for
MOX fuel fabrication. Thus the use of fixed enrichment for plutonium recycling
studies may bring similar bias for reprocessing rate calculation and plutonium
inventories in stocks. 

Bias of \gls{FF} use on plutonium global inventory and variation are estimated
by calculating the plutonium consumption ratio and consumption rate during
reactor operations. Those imply depletion calculations in fuel cycle studies
which add a new level uncertainties to our estimations. In the case of
\gls{SFR}, conclusions are tide with the reactor breeding ratio. For break-even
reactors, \gls{FF} does not bring any bias on the plutonium production of
incineration whereas for plutonium burner or breeder reactor, differences on
breeding ratio are in between $0.02$ and $0.05$. For PWR, two tendencies are
observed : ANICCA and TR\_EVOL shows respective typical biases of 59\% and 71\%
on the plutonium incineration ratio whereas CLASS and CYCLUS shows respective
biases of 12\% and 8\%. This burning ratio is directly linked to the global
inventory of plutonium in the fuel cycle. ANICCA and TR\_EVOL suggest that a
fixed fraction for plutonium enrichment in fresh MOX fuel may bring consequent
uncertainties on plutonium calculations. It should be pointed out that the
plutonium isotopic composition range is very wide and ANICCA and TR\_EVOL shows
some evolution close plutonium iso-breeding. This may be explained by the fact
that ANICCA and TR\_EVOL are used here for depletion calculations outside of
their validity domain.        

All the results presented here aims to estimate biases induced by a \gls{FF} for
plutonium enrichment for fresh-fuel composition. They don't give any clues about
\gls{FLM} accuracy and relevance. To qualify \gls{FLM}, benchmarks and
comparison with full core calculations are needed. Those are out of the FIT
project scope and are the responsibility of each software developers. The
observable treated here is linked to plutonium inventories for reprocessing
studies. \gls{FF} induced bias on other observables for other applications
should be assessed in the future within other exercises designed on purpose.   

\section{acknowledgment}
