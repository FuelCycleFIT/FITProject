\section{Conclusions and perspectives}

This paper presents the FIT project that aims to test different fuel cycle simulators functionalities for confidence estimation. Today numerous software for fuel cycle studies are developed by different institutions for different type of studies and different goals. Fuel cycle simulators can be used to assess different questions that requires different level of precision. To increase the confidence level of studies, all institutions are willing to increase the level of complexity of the software even if it does not necessary improve the precision of the calculations. 

The FIT project gathers several institutions worldwide to face the dilemma between fuel cycle complexity, confidence level of the produced output and code development needed to achieve the wanted precision. The FIT project approach consists in testing code functionality by an intra-code comparison of calculation with and without the feature. The FIT project is not a benchmark and there is no code to code comparison. The success of the FIT project relies on the number of different codes that are implies in the functionality test. The more software are implied, the stronger conclusions on the impact on functionalities are.   

With this methodology, the FIT project could give informations on which functionalities are required according to the question that needs to be answered and the precision level to reach.

The work presented here is dedicated fuel cycle simulators ability to adapt fresh fuel compositions to the available material. Two treatments are considered: a fixed enrichment for fresh fuel, whatever the isotopic composition or an enrichment calculation to adapt the fresh fuel composition in accordance to the isotopic composition. The first exercise treated here focuses on potentiality studies to recycle plutonium and the estimators built are consistent with this kind of studies.  

In agreement with the FIT philosophy all calculations were performed with a fixed fraction and with the use of a model that calculate fissile enrichment for fresh fuels. No code to code comparison are done here but each code is tested with and without a fixed fraction. The importance of adapting fresh fuel compositions is tested here for 3 main questions : the global plutonium inventory estimation, the global plutonium inventory evolution rate and the local plutonium inventory in spent fuel to be reprocessed. 4 different codes are tested for PWR (CLASS, ANICCA, TR\_EVOL and CYCLUS) and 3 for SFR (CLASS, TR\_EVOL and JOSETTE). 

Results shows that the FF calculations leads to bias for the BOC plutonium enrichment between 21\% and 40\% compared to FLM calculations for PWR. Biases for SFR calculation are observed between ZZ\% and TT\%. The BOC plutonium enrichment is directly related to the amount of spent fuel reprocessing fuel for MOX fuel fabrication. Then the use of fixed enrichment for plutonium recycling studies may bring similar bias for reprocessing rate calculation and plutonium inventories in stocks. 

Bias of FF use on global inventory and global evolution rate are estimated by calculating the plutonium consumption ratio and consumption rate during reactor operations. Those imply depletion calculations in fuel cycle studies which add uncertainties to our estimations. In the case of SFR, conclusions depends on reactor breeding ratio. For break-even reactors, FF does not bring any bias on the plutonium production of incineration whereas for plutonium burner or breeder reactor, biases on breeding ratio are in between XX\% and YY\%. For PWR, two tendencies are observed : ANICCA and TR\_EVOL shows respective typical biases of 59\% and 71\% on the plutonium incineration ratio whereas CLASS and CYCLUS shows respective biases of 12\% and 8\%. This burning ratio is directly linked to the global inventory of plutonium in the fuel cycle. ANICCA and TR\_EVOL suggest that a fixed fraction for plutonium enrichment in fresh MOX fuel may bring consequent uncertainties on plutonium calculations. It should be pointed out that the plutonium isotopic composition range is very wide and ANICCA and TR\_EVOL shows some evolution close plutonium iso-breeding. This may be explained by the fact that ANICCA and TR\_EVOL are used here for depletion calculations outside of their validity domain.        

All the results presented here aims to estimate biases induced by a FF for plutonium enrichment for fresh-fuel composition. They don't give any clues about FLM accuracy and relevance. To qualify FLM, benchmarks and comparison with full core calculations are needed. Those are out of the FIT project scope and are the responsibility of each software developers. The observable treated here is linked to plutonium inventories for reprocessing studies. FF induced bias on other observables for other applications should be assessed in the future within other exercises designed on purpose.   

\section{acknowledgment}