\section{Conclusions and Perspectives}
\label{sec:conclusion}

Substantial resources have been invested in a variety of nuclear fuel cycle Simulators, whether to serve specific purposes, or as they expanded to more general purposes.
However, confidence in each new analysis remains fragile because of a lack of generic methodology to develop confidence in such tools.
Fuel cycle simulators can be used to answer different questions that requires different level of precision.
To increase the confidence level of studies, all institutions are willing to increase the level of complexity of the software even if it does not necessary improve the precision of the calculations.

This paper presents the FIT project that provides a rigorous scientific method allowing researchers, developers and analysts to determine whether a feature, a functionality or a model refinement is required to perform a nuclear fuel cycle study within a certain precision goal.

The FIT project is not a benchmark.
The goal is not inter-simulators but intra-simulator comparison.
The impact of each functionality is assessed by comparing the simulator output from the same simulator, with and without the functionality to test.
Outputs of interest are not physical quantities observable in the fuel cycle but the deviation between the result with and without the functionality.
The strength of this project relies on the diversity of simulators used to solve the different problems that make it possible to reach nearly simulator-agnostic conclusion.
Following this principle, the FIT project will provide information about which functionalities are required according to the question that needs to be answered and the precision level to reach.

The work presented here is dedicated fuel cycle simulators ability to vary fresh fuel compositions according to the available materials.
Two treatments have been considered: a fixed enrichment for fresh fuels, regardless of the isotopic composition, and a model to adjust the fresh fuel composition in accordance with the isotopic composition.
In this first exercise, the output estimators focus on availability of the plutonium in the fuel cycle, so will probably be more valuable for fuel cycle study involving plutonium recycling.

In agreement with the FIT project’s principles, all calculations were performed with a fixed fraction and with the use of a model that calculate fissile enrichment for fresh fuels.
Some inter-software comparison have been performed but each code is tested against itself. \gls{FF} and \gls{FLM} have been compared on 2 different reactors technologies, across 5 different simulators, (CLASS, ANICCA, TR\_EVOL and CYCLUS) for the \gls{PWR} and (CLASS, TR\_EVOL, JOSETTE) for the \gls{SFR}.
The importance of adjusting fresh fuel compositions have been tested around 3 main questions:

\begin{itemize}
    \item the global plutonium inventory estimation 
    \item its evolution rate,
    \item the local plutonium inventories.
\end{itemize}
Results show large differences between \gls{FF} calculations and \gls{FLM} calculations: typically between $21\%$ and $40\%$ for the plutonium fraction in fresh \gls{PWR}-\gls{MOX} fuel.
For \gls{SFR} \gls{MOX} fuel, typical differences observed are between $9\%$ and $13\%$ depending on the simulators.
The \gls{BOC} plutonium enrichment is directly related to the amount of spent fuel reprocessing fuel for MOX fuel fabrication.
Thus, the use of \gls{FF} for plutonium recycling studies may bring similar bias for reprocessing rate calculation and plutonium inventories in stocks.

Bias due to \gls{FF} use on plutonium global inventory and variation are estimated by calculating the plutonium consumption ratio and consumption rate during reactor operations.
It implies depletion calculations in fuel cycle studies, which add a new level of uncertainties to our estimations.
For \gls{SFR}, conclusions are tied with the reactor breeding ratio.
With break-even reactors, \gls{FF} bring a small bias with only 2\% typical deviation on the plutonium production or incineration.
With burner or breeder reactors, typical difference on breeding ratio is a little higher, around 5\%.
For \gls{PWR}, two tendencies are observed: ANICCA and TR\_EVOL shows respective typical biases of 59\% and 71\% on the plutonium incineration ratio whereas CLASS and CYCLUS shows respective biases of 12\% and 8\%.
This burning ratio is directly linked to the global inventory of plutonium in the fuel cycle.
ANICCA and TR\_EVOL suggest that a fixed fraction for plutonium enrichment in fresh MOX fuel may bring consequent uncertainties on plutonium calculations.
It should be pointed out that the plutonium isotopic composition range is very wide and ANICCA and TR\_EVOL shows some evolution close to plutonium iso-breeding.
This may be explained by the fact that ANICCA and TR\_EVOL are used here for depletion calculations outside of their validity domain.

All the results presented here aims to estimate biases induced by a \gls{FF} for plutonium enrichment for fresh-fuel composition.
They don't give any clues about \gls{FLM} accuracy and relevance.
To qualify \gls{FLM}, benchmarks and comparison with full core calculations are needed.
Those are out of the FIT project scope and are the responsibility of each software developers.
The estimators treated here are linked to plutonium inventories for reprocessing studies.
Whether \gls{FF} induces bias on other output of use for other applications should be assessed in the future within other exercises designed on purpose.

\section{Acknowledgment}
Thank you Nico for the organization and management of the FIT project.