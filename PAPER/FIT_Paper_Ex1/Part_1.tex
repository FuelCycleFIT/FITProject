Since the 1990's, many different nuclear fuel cycle simulators have been
developed by several institutions.

A fuel cycle simulator aims to model an entire fleet of nuclear 
facilities such as nuclear reactors, fuel fabrication plants, reprocessing
plants, cooling pools, and waste repositories. These tools help to identify drivers
and interactions between parameters in the fuel cycle. They implement physics models
for key points in the cycle such as fuel enrichment, fabrication, and
depletions, with various levels of complexity. 

Fuel cycle simulators are used worldwide for wide range of applications:
optimizing industrial operation of an existing nuclear fleet, assessing
the future of nuclear energy, providing valuable information to political 
decision-makers, and evaluating/verifying the operation of an existing
nuclear fleet by national or international safety authorities. Moreover, these
tools are used for Research and Development training as access point to key
data related to the fuel cycle.

The myriad institutions that develop and use fuel cycle simulators pursue myriad
goals. Consequently, software development decisions (e.g. fidelity) are often 
made in accordance with the institution's simulation goals. To improve
confidence in the results, institutions may be tempted to increase the complexity
of their software even if this complexity might not be necessary.

As an example, neutron and gamma dose calculations require the precise
knowledge of each material isotopic composition in each facility whereas a
uranium consumption calculation does not require the same degree of detail. As a
consequence, some software functionality may not be necessary regarding the
technical question the code assesses: solving a given technical question
associated with a targeted precision will required a limited set of
functionality. Knowing the importance of each will help users choose an
appropriate software tool or may guide future code development to solve a specific
question. Also, some technical issues are assessed by numerous studies performed
with different software and it is often difficult to compare them. Knowing the
impact of functionalities on different simulation outputs helps to estimate the
level of confidence of the different fuel cycle studies.

The \gls{FIT} Project has been conceived to understand
the circumstances under which the choice of algorithm and/or model influences
the conclusions that one might draw from a fuel cycle simulation. The
project aims to characterize the relationship between model fidelity and desired 
confidence level in the context of many differing research questions.
Among functionalities of interest, this first \gls{FIT} focuses on the ability of
fuel cycle software to build fresh fuel in accordance with the available material 
and with associated reactor requirements.


Section \ref{sec:framework} describes the \gls{FIT} project, its philosophy, the
participants, and associated simulators. It explains how the \gls{FIT} project 
is not a traditional benchmark. That is, \gls{FIT}does not aim to do 
inter-simulator comparison, but focuses on intra-simulator comparison, 
evaluating differences between simulation results produced by the same 
simulator, enabling and disabling isolated features. In order to build 
confidence in the conclusion, such comparison will be done across multiple 
simulators. Section \ref{sec:framework} ends with the description of the 
particular feature tested in this work, the fuel loading management model. 
Section \ref{sec:exercise} then presents the numerical experiment used to test 
this particular feature, including simulation input parameter descriptions and 
other technical specifics necessary to perform the test. Additionally, this 
section details output metrics which quantify the impact of the use of fuel 
loading management in fuel cycle studies. Next, Section \ref{sec:results} is 
dedicated to the discussion of results acheived by each software tool involved 
in this first exercise. Finally, in Section \ref{sec:conclusion} some 
conclusions are drawn.
