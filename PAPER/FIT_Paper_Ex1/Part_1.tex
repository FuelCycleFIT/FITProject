Since the 1990's, many different nuclear \gls{FCS}s have been
developed by several institutions.

A \gls{FCS} aims to models an entire nuclear fleet including main facilities,
such as nuclear reactors, fuel fabrication plants, reprocessing plants, cooling
pool and/or waste disposal. Those tools help to identify drivers and
interactions between parameters in fuel cycle. They implement physic
models for different key points of the cycle such as fuel fabrication or burn up
depletions, with various level of complexicity. 

\gls{FCS} are used worldwide for wide range of applications: optimisation of the
industrial operation of a existing nuclear fleet, assessing the future of
nuclear energy and providing valuable informations required for the political
decision, developement of tools dedicated to non-proliferation treaties
verification. Moreover, those tools are used for Research and Development
training as access point to key datas related to the fuel cycle. 

The different institutions that use (and develop) fuel cycle simulators pursue
different goals. Consequently, their computational needs vary in term of
complexity and the level of complexity of each software pieces has to be in
adequation with regards to their different simulations requirements. To improve
confidence in the results, institutions are tempted to increase the complexity
of their software even if this complexity might not be necessary.

As an example, the neutron and gamma doses calculation requires the precise
knowledge of each material isotopic composition in each facilities whereas
uranium consumption calculation does not require the same degree of detail. As a
consequence, some software functionality may not be necessary regarding the
technical question the code assesses: solving a given technical question
associated with a targeted precision will required a limited set of
functionality complexity. Knowing the importance of each ones helps to choose an
appropriate software or to engage code developments to solve a specific
question. Also, some technical issues are assessed by numerous studies performed
with different software and it is often difficult to compare them. Knowing the
impact of functionality on different simulation outputs helps to rank them by
level of confidence.

The FIT (Functionality Isolation Test) Project has been conceived to understand
the circumstances under which the choice of algorithm and/or model influences
the conclusions that one might draw from such a fuel cycle simulation. Ths
project aims to determine the minimum level of details in fuel cycle simulations
required as a function of the study and the wanted confidence level. Among
different functionality of interest, this first work focuses on the ability of
fuel cycle software to build fresh fuel regarding to the available material for
fuel fabrication and the reactors requirements.

The first part of the paper describes the FIT project, its philosophy and the
participants and associated simulators. It explains why the FIT project is not a
tradictionnal benchmark and does not aim to do inter-simulator comparison, but
focuces on intra-simulators comparison, evaluating differences between
simulation results, produced by the same simulator, enabling and disabling  the
features to test. In order to build confidence in the conclusion, such
comparison will be done accross multiple simulators. This part ends with the
description of the particular feature tested in this work which is the fuel
loading management. The second part presents the design of experiment used to
test this particular feature: input simulation descriptions, technical
specificities and finally output metrics used to quantify the impact of the use
of fuel loading management in fuel cycle studies are detailed. Finally, the
third part is dedicated to the different results of each software involved in
this first exercise and some conclusion are withdrawn.
