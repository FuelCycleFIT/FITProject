% coucou

\section{Results}

This section presents the different calculation results obtained by the
different simulators and the estimators distribution defined in the previous
section.  The assumption made here is to suppose that introducing a physic model
like a \gls{FLM} into a fuel cycle simulation represents better the reality than
no model like a \gls{FF} for all plutonium fresh compositions. Our calculations
don't aim to validate any model nor to compare \gls{FLM}s between them and
select one. It tries to assess the need of \gls{FLM} for some local or global
conclusion in fuel cycle studies by estimating bias on elementary calculations.
Actually, our work aims to give conclusions about the usage of \gls{FF} model.   

\subsection{Pressurized Water Reactor}
\subsubsection{Output analyses}

Figure~\ref{fig:PWR_MOX_FLM_Pu} presents the plutonium fraction at \gls{BOC} predicted
by each \gls{FLM} and the plutonium fraction at EOC deduced by each software. As all
FLM are different, the \gls{BOC} plutonium fractions differ from a software from
another. The widest prediction is given by the CLASS code that predicts plutonium
fraction from 4\% until more that 15\%. This range is a direct consequence of
the plutonium sampling used for this work. 15\% is clearly unrealistic but some
of the plutonium isotopic composition sampled are not either as it contains a low
amount of fissile. That's why the \gls{FLM}s may reach such high values.    

\begin{figure}[h]
	\begin{center}
		\includegraphics[width = 0.99\textwidth]{../../Feature_1/RAW_DATA/FIG/PWR_MOX_FLM_Pu.pdf}
		\caption{Code outputs for \gls{PWR} scenario's calculations}
		\label{fig:PWR_MOX_FLM_Pu}
	\end{center}
\end{figure}

The EOC plutonium fraction is clearly shifted to the lower values, proving that
plutonium is consumed during irradiation like it should be in \gls{PWR}s. For ANICCA
and for TrEVOL however, some calculations show configurations where the
plutonium fraction is higher at EOC than it is at \gls{BOC}. This unrealistic
result may be explained by the very wide plutonium isotopic composition range.
ANICCA and TrEVOL were not conceived to simulate evolution of such a wide range
of composition. This strange behavior may be attributed to the depletion
calculation of those extreme fuels. Results about estimator 2 and 3 evaluating
plutonium consumption should then be handled carefully with TrEVOL and ANICCA
results.   


\subsubsection{Estimator's calculation}

Figure~\ref{fig:Est1_PWR}, figure~\ref{fig:Est2_PWR} and
figure~\ref{fig:Est3_PWR} represent, for the \gls{PWR}, the distribution of the
different estimators defined in the section\ref{subsec:estimator}. 

\paragraph{Plutonium fraction at \gls{BOC}}

Estimator 1 aims to quantify bias introduced by the use of a \gls{FF} model on
the plutonium enrichment calculation for fresh fuel. It measures the amount of
plutonium used for \gls{MOX} fuel fabrication. An important positive bias means
that the \gls{FF} model underestimate the mass of spent fuel to be processed for
the MOX fuel fabrication. As the \gls{FF} was tuned on a standard plutonium
composition, in the middle of the isotopic space used for sampling,
Figure~\ref{fig:Est1_PWR} presents some histograms almost centered on 0.
        

\begin{figure}[h]
	\begin{center}
		\includegraphics[width = 0.99\textwidth]{../../Feature_1/RAW_DATA/FIG/PWR_MOX_Estimator_1.pdf}
		\caption{Estimator 1 for \gls{PWR} calculated with ANICCA, CLASS, CYCLUS and TrEVOL}
		\label{fig:Est1_PWR}
	\end{center}
\end{figure}

The standard deviation of the distribution is also a relevant quantity as it
quantifies the dispersion of the calculation bias (i.e. i\textit{estimator 1}).
They are presented in table~\ref{table:Est1_PWR}. A small standard deviation
implies a narrow distribution and means small calculation biases due to the use
of \gls{FF} model. As we can see on the plot, none of the used software for this
work shows a small standard deviation. Meaning that, within the range of the
sampled plutonium composition, the use of a \gls{FF} model induce ``large'' bias
on the fresh \gls{MOX} fuel plutonium fraction, leading to ``large`` bias on the
mass of reprocessed spent fuel for the \gls{MOX} fuel fabrication.

\begin{table}[h]
	\begin{center}
		\begin{tabular}{|c||c||c||c|}
			\hline 
				CLASS & ANICCA & TR\_EVOL & CYCLUS \\
			\hline
				XX & YY & ZZ & TT \\
		\end{tabular}
	\end{center}
	\label{table:Est1_PWR}
\end{table}

\paragraph{Ratio between plutonium consumption and plutonium at \gls{BOC}}

Estimator 2 aims to quantify the amount of plutonium consumption regarding the
plutonium mass at \gls{BOC}. It measures the proportion of plutonium burnt
during irradiation. Estimator 2 distribution gives an evaluation of the total
inventory precision estimation, regardless the location of the plutonium.
Figure~\ref{fig:Est2_PWR} represents the different distributions of this estimator
for the 4 codes used in this work. There are two trends in this plot: CYCLUS and
CLASS shows limited bias (with a standard deviation of approximately XX as it
can be seen in table~\ref{table:Est2_PWR}), whereas TR\_EVOL and ANICCA
calculates very strong biases. From this different behaviors, it is impossible
to conclude on the \gls{FLM} relevance for total inventory estimation. A limited bias
(as seen with CYCLUS and CLASS) means that the use of a \gls{FF} for plutonium
enrichment at \gls{BOC} may be sufficient for global inventory estimation. On the
opposite, results from ANICCA and TR\_EVOL tends to show that a \gls{FLM} is
necessary. 

\begin{figure}[h]
	\begin{center}
		\includegraphics[width = 0.99\textwidth]{../../Feature_1/RAW_DATA/FIG/PWR_MOX_Estimator_2.pdf}
		\caption{Estimator 2 for \gls{PWR} calculated with CLASS, CYCLUS and TrEVOL}
		\label{fig:Est2_PWR}
	\end{center}
\end{figure}

\begin{table}[h]
	\begin{center}
		\begin{tabular}{|c||c||c||c|}
			\hline 
				CLASS & ANICCA & TR\_EVOL & CYCLUS \\
			\hline
				XX & YY & ZZ & TT \\
		\end{tabular}
	\end{center}
	\label{table:Est2_PWR}
\end{table}

It has to be pointed out that this estimator (as for estimator 3 shown in the
next paragraph) needs a depletion calculation.  Figure~\ref{fig:PWR_MOX_FLM_Pu}
shows that this depletion calculation is questionable for some plutonium
isotopic compositions with ANICCA and TR\_EVOL.  Those two software may be used
beyond their validity domain and this might be reflected in the very wide
distribution of Estimator 2. The depletion calculation, inaccurate for
composition outside of the validity domain, adds some biases to the one brought
by the \gls{FF} model uses. The conclusions made with ANICCA and TR\_EVOL has
then to be consider with caution.

\paragraph{Plutonium consumption rate}

Estimator 3 measures biases on the plutonium consumption rate calculation. As
so, it measures the biases induced by a \gls{FF} for plutonium enrichment at
\gls{BOC} on a global scale. It aims to quantify the global inventory evolution
rate.  Figure~\ref{fig:Est3_PWR} represents estimator 3 for CLASS, ANICCA,
TR\_EVOL and CYCLUS. Similar conclusion can be drawn as previous section. CLASS
and CYCLUS are in agreement as shows that the impact of using a \gls{FF} may be
limited, introducing $10\%$ bias on the total plutonium consumption rate. On the
contrary ANICCA and TR\_EVOL are also in agreement showing a much larger bias.
The same caution of previous section should be pointed out: depletion
calculation of the latest software are out of validity domain adding some
uncertainty to the bias calculated by the use of \gls{FLM}.       

\begin{figure}[h]
	\begin{center}
		\includegraphics[width = 0.99\textwidth]{../../Feature_1/RAW_DATA/FIG/PWR_MOX_Estimator_3.pdf}
		\caption{Estimator 3 for \gls{PWR} calculated with CLASS, CYCLUS and TrEVOL}
		\label{fig:Est3_PWR}
	\end{center}
\end{figure}

\subsection{Fast Sodium cooled Reactor}
\subsubsection{Output analyses}

Figure~\ref{SFR_MOX_FLM_Pu} represents the plutonium enrichment distribution at
BOC (in black) and at EOC (in red) simulated with CLASS, JOSETTE and TR\_EVOL.
The different reactor models used for the different software explain the
different behavior. CLASS simulates a breeder whereas TR\_EVOL a simulate a
burner and JOSETTE a break-even reactor. Plutonium mass evolution is then
different for all those software. The purpose of this paper is not to compare
software but to compare the use or not for each of those software. The fact that
the reactor models shows different behavior reinforces conclusions drawn for the
importance of \gls{FLM} in fuel cycle simulators.     

\begin{figure}[h]
	\begin{center}
		\includegraphics[width = 0.99\textwidth]{../../Feature_1/RAW_DATA/FIG/SFR_MOX_FLM_Pu.pdf}
		\caption{Code outputs for \gls{PWR} scenario's calculations}
		\label{fig:SFR_MOX_FLM_Pu}
	\end{center}
\end{figure}

\subsubsection{Estimator's calculation}
\paragraph{Plutonium fraction at \gls{BOC}}

Figure~\ref{fig:Est1_SFR} represents the estimator 1 calculated for Sodium
Cooled Fast Reactors calculated with JOSETTE, TrEVOL and CLASS. Like for
\gls{PWR}, it shows the relative difference of plutonium enrichment with the use
of a \gls{FLM} in regards to a \gls{FF}. Standard deviations of the difference
distribution is given in Table~\ref{table:Est1Dev_SFR} for the different codes.
All codes are in agreement even if CLASS slightly underestimate the \gls{FF}
impact on the plutonium needed for a SFR. The typical bias produced by the use
of a \gls{FF} is smaller than 25\% and is much lower than for \gls{PWR}.   

\begin{figure}[h]
	\begin{center}
		\includegraphics[width = 0.99\textwidth]{../../Feature_1/RAW_DATA/FIG/SFR_Estimator_1.pdf}
		\caption{Estimator 1 for SFR calculated with JOSETTE, TrEVOL, and CLASS}
		\label{fig:Est1_SFR}
	\end{center}
\end{figure}

\begin{table}[h]
	\begin{center}
		\begin{tabular}{|c||c||c|}
			\hline 
				JOSETTE & TrEVOL & CLASS \\
			\hline
				XX & YY & ZZ \\
		\end{tabular}
	\end{center}
	\label{table:Est1Dev_SFR}
\end{table}

The estimation of spent fuel mass that have to be reprocessed for SFR fresh fuel
composition using a \gls{FF} for plutonium enrichment is then estimated with a bias
smaller than 10\% no matter the SFR behavior (breeder, burner or break-even).  

\paragraph{Ratio between plutonium consumption and plutonium at \gls{BOC}}

Estimator 2.b aims to calculate the plutonium production or consumption on the
global level of the fleet. It estimates the absolute difference on breeding
ratio when simulation are made with a \gls{FF} or with a \gls{FLM}. CLASS and TR\_EVOL shows
similar results and calculate biases induced by the \gls{FF} method up to 0.1. JOSETTE
calculations shows a narrower distribution, probably due to the fact that SFR
simulated with JOSETTE are all break-even.   

\begin{figure}[h]
	\begin{center}
		\includegraphics[width = 0.99\textwidth]{../../Feature_1/RAW_DATA/FIG/SFR_Estimator_2b.pdf}
		\caption{Estimator 2.b for SFR calculated with JOSETTE, TrEVOL, and CLASS}
		\label{fig:Est2_SFR}
	\end{center}
\end{figure}
